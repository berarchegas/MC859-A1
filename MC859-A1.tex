\documentclass{article}
\usepackage[utf8]{inputenc}
\usepackage{amsmath}
\usepackage{graphicx} % Required for inserting images

\usepackage[draft]{hyperref}
\usepackage{xspace}

\newcommand{\outDegree}{\ensuremath{\mathit{outDegree}}\xspace}
\newcommand{\inDegree}{\ensuremath{\mathit{inDegree}}\xspace}

\usepackage[T1]{fontenc}
\usepackage[utf8]{inputenc}
\usepackage[brazil]{babel}
\usepackage{setspace}
\usepackage{amsthm}

\usepackage{cfr-lm}
\usepackage{microtype}

\newtheorem{lemma}{Lema}[section]

\usepackage{array, makecell, multirow,
            ltablex}            % <--- doesn't change anything, 
                                % however for keep X column features
                                % you should add (before table) \keepXColumns 
\newcolumntype{B}{>{\global\let\currentrowstyle\relax}}
\newcolumntype{^}{>{\currentrowstyle}}
\newcommand{\rowstyle}[1]{\gdef\currentrowstyle{#1}%
                          #1\ignorespaces}
\newcolumntype{C}{>{\centering\arraybackslash}X}   % <-- new
\usepackage{siunitx}

\title{MC$859$ - A$1$}
\author{Andreas Cisi Ramos $246932$ \\
Bernardo Panka Archegas $246970$ \\
Bruno Amaral Teixeira de Freitas $246983$ \\}
\date{}

\usepackage{listings}
\usepackage{color}

\definecolor{dkgreen}{rgb}{0,0.6,0}
\definecolor{gray}{rgb}{0.5,0.5,0.5}
\definecolor{mauve}{rgb}{0.58,0,0.82}

\lstset{frame=tb,
  language=C,
  aboveskip=3mm,
  belowskip=3mm,
  showstringspaces=false,
  columns=flexible,
  basicstyle={\small\ttfamily},
  numbers=none,
  numberstyle=\tiny\color{gray},
  keywordstyle=\color{blue},
  commentstyle=\color{dkgreen},
  stringstyle=\color{mauve},
  breaklines=true,
  breakatwhitespace=true,
  tabsize=3
}

\begin{document}
\doublespacing
\maketitle

\section{Introdução}

Nosso objetivo é relatar o processo do nosso grupo de aplicação da meta heurística do GRASP ao problema MAX-KQBF, que envolve maximizar uma função binária quadrática sob uma restrição de capacidade. O desafio é encontrar um conjunto de variáveis cuja soma dos pesos não exceda um limite dado, enquanto se maximiza o valor da função.

\section{Descrição do Problema}

O problema MAX-KQBF pode ser formalmente descrito como segue: Dada uma capacidade $W \in \mathbb{R}$ e um peso $w_i \in \mathbb{R}$ associado a cada variável $x_i$, deseja-se maximizar uma função binária quadrática sujeita à condição de que a soma dos pesos das variáveis selecionadas não exceda a capacidade $W$. Este problema é uma extensão do clássico problema da mochila, incorporando elementos de interações entre pares de itens através da função binária quadrática.

Esse problema já é NP-difícil mesmo se removermos a restrição de capacidade. Nesse caso, estamos lidando com uma versão ainda mais restrita.

\section{GRASP e nossa implementação}

O GRASP (Greedy Randomized Adaptive Search Procedure) é uma meta-heurística para encontrar soluções aproximadas para problemas de otimização difícil. Nossa implementação do GRASP para o MAX-KQBF consiste em duas fases principais: construção e busca local. Na fase de construção, geramos soluções iniciais de forma gulosa e aleatória, garantindo diversidade enquanto mantemos a qualidade das soluções. Na fase de busca local, refinamos essas soluções explorando seus vizinhos até alcançar um ótimo local. A alternância entre construção gulosa e busca local permite ao GRASP explorar eficientemente o espaço de soluções, balanceando construções aleatórias e determinísticas.

\subsection{Fase de construção}

Para essa parte utilizamos três opções de algoritmos: a versão padrão, Greedy Randomized Construction, e dois métodos alternativos, Sampled Greedy Construction e Random Plus Greedy Construction.

Na fase de construção padrão o conjunto de elementos candidatos é formado por todos os elementos do conjunto base que podem ser incorporados à solução parcial sem comprometer a construção de uma solução viável com os elementos restantes. A seleção do próximo elemento é baseada em uma função gulosa que avalia o custo incremental da sua inclusão na solução em construção. Os melhores elementos formam uma lista restrita de candidatos, de onde um é escolhido aleatoriamente para inclusão na solução parcial. Após a escolha, a lista é atualizada e os custos incrementais são recalculados. Esse processo continua até não haver mais candidatos disponíveis.

Para o método alternativo Random Plus Greedy Construction  durante os primeiros \textit{p} passos da construção, o algoritmo introduz aleatoriedade na seleção dos elementos para construir uma solução parcial. Logo, o algoritmo escolhe os elementos candidatos de forma mais aleatória. Após os primeiros \textit{p} passos, o algoritmo muda para uma abordagem puramente gulosa. Nesta fase, a aleatoriedade é eliminada e o algoritmo seleciona os elementos candidatos com base em critérios gulosos, buscando otimizar a solução parcial de forma determinística.Pode-se controlar o equilíbrio entre a parte gulosa e aleatória alterando o valor do parâmetro \textit{p}. Valores maiores de \textit{p} estão associados a soluções mais aleatórias, enquanto valores menores resultam em soluções mais gulosas.

Para o método alterativo Sampled Greedy assim como o método anterior também é  controlado por um parâmetro \textit{p}.Em cada etapa do processo de construção, o algoritmo constrói uma lista de candidatos restritos, amostrando aleatoriamente \textit{p} elementos do conjunto de candidatos C, no caso de \textit{p} ser maior que o número de elementos do conjunto, então todos os elementos do conjunto são selecionados. Cada elemento da lista restrita é avaliado pela função gulosa. O elemento com o menor valor dessa avaliação é adicionado à solução parcial. Esses dois passos são repetidos até que não haja mais elementos candidatos.O equilíbrio entre a parte gulosa e aleatória também pode ser controlado alterando o valor do parâmetro \textit{p}, ou seja, o número de elementos candidatos que são amostrados.



\subsection{Fase de busca local}

Para a fase de busca local, utilizamos 2 estratégias: best-improving e first-improving. No caso da primeira, todos os vizinhos são investigados e a solução atual é trocada pelo melhor vizinho. Já para a segunda, a solução é trocada pelo primeiro vizinho avaliado que constituir uma solução melhor.

Definimos vizinhança como qualquer solução que delete no máximo um elemento do conjunto e adicione novos elementos conforme possível, até a restrição da capacidade da mochila não permitir mais.

\section{Resultados}

Considerando uma configuração padrão do GRASP - com first-improving na etapa de busca local - tomando $\alpha$ igual a 10\%, temos os seguintes resultados na Tabela 1:

\begin{table}[htbp]
\centering
\begin{tabular}{|c|c|}
\hline
\textbf{Caso de teste} & \textbf{Resultado} \\
\hline
Caso de teste 1 & 84 \\
Caso de teste 2 & 233 \\
Caso de teste 3 & 486\\
Caso de teste 4 &  - \\
Caso de teste 5 &  - \\
Caso de teste 6 & 2591 \\
Caso de teste 7 & 7195 \\

\hline
\end{tabular}
\caption{Resultados para a versão padrão do GRASP com $\alpha$ = 10\%}
\label{tab:minha_tabela}
\end{table}

Percebe-se que o algoritmo não foi capaz de produzir resultado para os casos de teste 4 e 5.

Considerando novamente a versão padrão do GRASP, mas tomando $\alpha$ igual a 5\%, temos os seguintes resultados na Tabela 2:

\begin{table}[htbp]
\centering
\begin{tabular}{|c|c|}
\hline
\textbf{Caso de teste} & \textbf{Resultado} \\
\hline
Caso de teste 1 & 84 \\
Caso de teste 2 & 233 \\
Caso de teste 3 & 486\\
Caso de teste 4 &  - \\
Caso de teste 5 & 877 \\
Caso de teste 6 & 2550 \\
Caso de teste 7 & 6895 \\

\hline
\end{tabular}
\caption{Resultados para a versão padrão do GRASP com $\alpha$ = 5\%}
\label{tab:minha_tabela}
\end{table}

Nota-se que o algoritmo não foi capaz de produzir resultados para o teste 4.
    
Considerando a versão padrão do GRASP utilizando best-improving na etapa de busca local, com $\alpha$ igual a 5\%, temos os seguintes resultados na Tabela 3:

\begin{table}[htbp]
\centering
\begin{tabular}{|c|c|}
\hline
\textbf{Caso de teste} & \textbf{Resultado} \\
\hline
Caso de teste 1 & 84 \\
Caso de teste 2 & 230 \\
Caso de teste 3 & 461 \\
Caso de teste 4 & 760 \\
Caso de teste 5 & - \\
Caso de teste 6 & 2652 \\
Caso de teste 7 & 7059 \\

\hline
\end{tabular}
\caption{Resultados para a versão padrão do GRASP com best-improving na busca local e com $\alpha$ = 5\%}
\label{tab:minha_tabela}
\end{table}

Considerando o GRASP com Sampled Greed Construction utilizando first-improving na etapa de busca local e com p = 30\% do número de entradas, temos os seguintes resultados na Tabela 4:

\begin{table}[htbp]
\centering
\begin{tabular}{|c|c|}
\hline
\textbf{Caso de teste} & \textbf{Resultado} \\
\hline
Caso de teste 1 & 84 \\
Caso de teste 2 & 235 \\
Caso de teste 3 & 486 \\
Caso de teste 4 & - \\
Caso de teste 5 & 872 \\
Caso de teste 6 & 2572 \\
Caso de teste 7 & 7055 \\

\hline
\end{tabular}
\caption{Resultados para o GRASP utilizando Sampled Greed Construction e first-improving na busca local}
\label{tab:minha_tabela}
\end{table}

Percebe-se que não obteve-se resultado para o caso de teste 5.

Considerando a versão do GRASP utilizando Random Plus Greedy Construction com first-improving na etapa de busca local, temos os seguintes resultados na Tabela 5:

\begin{table}[htbp]
\centering
\begin{tabular}{|c|c|}
\hline
\textbf{Caso de teste} & \textbf{Resultado} \\
\hline
Caso de teste 1 & 84 \\
Caso de teste 2 & 233 \\
Caso de teste 3 & 457 \\
Caso de teste 4 & 607 \\
Caso de teste 5 & 610 \\
Caso de teste 6 & 1566 \\
Caso de teste 7 & 3188 \\

\hline
\end{tabular}
\caption{Resultados para o GRASP utilizando Random Plus Greedy Construction com first-improving na etapa de busca local}
\label{tab:minha_tabela}
\end{table}

\section{Conclusão}

Primeiramente, ao comparar os resultados entre as configurações padrão do GRASP com $\alpha$ de 10\% e 5\%, observa-se que, em geral, a redução de $\alpha$ levou a resultados ligeiramente melhores, indicando que uma menor diversificação pode ter sido mais eficaz para explorar o espaço de busca. Além disso, ao utilizar o método best-improving na etapa de busca local, observou-se uma melhoria consistente nos resultados, sugerindo que essa abordagem pode ser mais eficiente na busca por soluções ótimas ou próximas ao ótimo. Por outro lado, ao comparar os métodos de construção de solução, como Sampled Greed Construction e Random Plus Greedy Construction, observa-se que ambos tiveram desempenhos semelhantes em termos de resultados, indicando que a estratégia de construção inicial pode não ter um impacto tão significativo quanto a etapa de busca local na qualidade das soluções encontradas. Ajustes adicionais nos parâmetros e na configuração do algoritmo podem ser necessários para otimizar ainda mais o desempenho do GRASP em uma ampla gama de casos de teste.

Fazendo uma comparação com os valores das soluções ótimas:

\begin{table}[htbp]
\centering
\begin{tabular}{|c|c|c|}
\hline
\textbf{Instância} & \textbf{|x|} & 
\textbf{MAX-KQBF (Z*)} \\
\hline
kqbf020 & 20 & [80,151] \\
kqbf040 & 40 & [275,429] \\
kqbf060 & 60 & [446,576] \\
kqbf080 & 80 & [729,1000] \\
kqbf100 & 100 & [851,1539] \\
kqbf200 & 200 & [3597,5826] \\
kqbf400 & 400 & [10846,16625] \\

\hline
\end{tabular}
\caption{Os nomes das instâncias,
suas dimensões e os intervalos nos quais os valores das soluções ótimas}
\label{tab:minha_tabela}
\end{table}

Comparando os resultados esperados com os valores obtidos, percebe-se que foi possível obter valores dentro do intervalo esperado para instâncias de até 100 elementos. Para os casos nos quais |x| = 200 ou |x| = 400, as nossas abordagens não foram capazes de atingir valores dentro do intervalo esperado.

Uma possível razão para isso talvez seja o método de construção de vizinhos para a etapa de busca local, uma vez que adicionamos novos elementos à mochila sem fazer uma análise da contribuição desses elementos para a função que deseja-se maximizar.

\end{document}
